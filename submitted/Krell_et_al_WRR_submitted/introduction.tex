In semiarid and arid regions of sub-Saharan Africa rising temperatures and shifting rainfall patterns are projected to negatively impact agricultural output \cite{Downing1997-we, Slingo2005-ms, muller2011climate, Branca2011-al}. Changes in rainfall associated with climate variability directly impact crop growth as storms are projected to become more intense with longer periods between rainfall events \cite{meehl2007global,donat2016more, harrison2019identifying, adloff-inreview}. The stochastic nature of rainfall during the growing season leaves crops susceptible to water stress during critical stages of development and can lead to crop failure \cite{sah2020impact, salgado2020physiological}. Small rainfed farms cultivated by single families on plots less than 5 hectares represent the most prevalent form of agriculture in sub-Saharan Africa and are particularly vulnerable to climate variability and change \cite{samberg2016subnational}. Because of their dependence on rainfed agriculture \cite{dinar2008climate}, smallholder livelihoods are susceptible to climate shocks that affect food prices \cite{ray2012recent}, variability in production and supply \cite{lobell2011climate, Slingo2005-ms}, and farmer incomes \cite{reidsma2010adaptation}. 

Farmers make a variety of choices before and during the growing season that impact their agricultural production and thus food security and livelihood. Cultivar choice is one of the most critical choices a smallholder makes \cite{Kalanda-Joshua2011-ot}. Because of the uncertainty associated with climate variability, farmer decision-making is becoming increasingly complex and uncertain at the expense of input use efficiency and profitability \cite{Hansen2011-bk, waldman2019cognitive, guido2020farmer}. Management options that were optimal for past or average climatic conditions may no longer be suited for increasingly common growing season weather. Additionally, traditional crop varieties may no longer be best suited for a farmer’s environment, which has led to the development of hybrid and fast growing varieties \cite{Smale2010-cv}. Given ongoing changes in rainfall patterns, farmers need to select cultivars well suited for their local context that can lead to the greatest payoff in terms of yield while also minimizing the risk of crop failure. To date, however, no modeling exercise to understand the effect of stochastic rainfall variability on maize yields for various cultivars exists for dryland environments in sub-Saharan Africa.  

To evaluate the impact of farmer decision-making and climate variability on agricultural outcomes, both field and modeling approaches have been employed (e.g. \citeA{Bharwani2005-vz, Ziervogel2005-qu, Roudier2014-qk, Vervoort2016-au, Wood2014-aq, Choi2015-ec}.) While field studies provide empirical evidence of environmental impacts on farmer outcomes, they can be limited to certain conditions, especially when panel data are absent \cite{patt2005effects, Hansen2011-bk} and are difficult to extrapolate to scenarios where the climate is changing. Alternatively, crop models can be useful when field data are unavailable, but such models can also be limited in applicability and need to be carefully parameterized, specifically for those parameters governing the stochastic nature of rainfall. In rainfed contexts, variability in inter- and intra-annual rainfall is closely linked to variability in production. Crop models and agronomic studies in general focus on annual, seasonal or monthly rainfall totals \cite{barron2003dry} and do not provide a much-needed evaluation of within-season variability, which has important implications for crop yields \cite{recha2012determination}. Dryland regions in particular necessitate careful modeling of rainfall patterns that are heterogeneous in space and time. An improved understanding of rainfall variability considers the temporal distribution of rainfall through analyses of the average amount of rain during rainfall events and the average length of time between successive events \cite{recha2012determination}.

In order to develop a better way to study these systems, two concerns must be addressed. First a more accurate consideration of rainfall dynamics in semiarid environments is needed. Considering the stochastic nature of rainfall rather than the seasonal averages is important in these systems where the frequency and duration of rainfall lead to important consequences for vegetation response \cite{Katul2007-tj, Porporato2002-uq}. Localized convective storms arrive in pulses that beget nonlinear vegetation response \cite{Katul2007-tj, Baudena2007-tg}. Second, in addition to considerations of the hydroclimatic environment, the representation of vegetation needs to be specific to a crop of interest. While researchers have separately undertaken modeling exercises to understand the impact of climate variability on crop growth and the stochastic nature of rainfall on vegetation structure, there have been fewer efforts to link stochastic rainfall dynamics to the probability of crop failure for staple crops such as maize. Specifically, understanding the influence of cultivar choice on the success of a crop has not been considered.

This study is motivated by the need to better understand the coupled dynamics of water and rainfed agricultural systems in dryland regions occupied by smallholder farmers. We address this need by presenting a model of stochastic seasonal soil water availability that evaluates the impacts of intra-seasonal rainfall variability on crop production in a smallholder agricultural system. This model is based on a previously explored stochastic soil water balance model \cite{laio2001plants, Laio2001-vb, Rodriguez-Iturbe2001-un, Porporato2001-ui} that simulates the interactions between soil, plants, and climate. The point-based model is nonspatial and determines daily growing and harvest season values of runoff, interception, leakage and evapotranspiration for a given soil type and cultivar. Rainfall is represented as a marked Poisson process expressed as the mean depth of daily rainfall and the mean probability of storm arrival, which forces the model at the daily time step.

We apply this model to a study site in central Kenya that exhibits a high degree of rainfall variability. Using a long-term daily precipitation dataset and a characteristic soil type for the region we estimate climate and soil parameters for the model. We use the stochastic soil water balance model to determine yield outcomes and the probability of crop failure, which are a function of plant water deficit (dynamic water stress). We compare and evaluate our model results for various maize varieties with late, medium and early harvesting periods. We aim to answer the following questions for our study area:
\begin{enumerate}
\item What do historical records (40+ years) indicate about inter-annual rainfall trends? 
\item How do varying maize cultivars moderate the effect of climate variability on changes in yields?
\item Has the likelihood of crop failure changed over an 80 year period?
\end{enumerate}

The following paper is organized as follows: We first describe our study site and the hydro-meteorological data necessary to apply the soil water balance model in the methods section. We then introduce our modeling framework and metrics for converting stress into yield. Our results demonstrate the impact of intra-seasonal rainfall variability on the seasonal water availability of maize varieties. We discuss these findings in the context of smallholder farmer decision-making and explore the implications of the results in the context of historical trends in rainfall and thus crop failure. We conclude with a discussion of model limitations and suggest additional research agendas appropriate for our proposed model.