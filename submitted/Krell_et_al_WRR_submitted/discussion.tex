\subsection{Rainfall trends and impacts on maize production}

Warming global temperatures directly cause inter-annual variability in rainfall---i.e., the shift towards fewer and more intense storms \cite{IPCC2007}---as observed at Jacobson Farm and other gauges in the field site (Table \ref{table:mk}). With air temperatures and atmospheric water vapor rising in East Africa as well as globally, extreme precipitation events are expected to increase \cite{trenberth2011changes,christiansen2007}. We find that while rainfall seasonal totals are not changing, the inter-storm arrival rate and average storm depth have changed in several sites in Laikipia, Meru and Nyeri counties.

When considering the extremes of the 80-year rainfall record (i.e. 1930s versus 2010s rainfall climatology), we see changes in cropping outcomes among maize farmers. Overall, we find a downward trend in yield and an upward trend in crop failure due to the pattern of decreased rainfall frequency and increased rainfall intensity. Because the inter-storm arrival rate has decreased by half and the average storm size has roughly doubled it is becoming more difficult to grow maize in these rainfed systems. Previous work has demonstrated the similar finding that the intensity and duration of individual rainfall events is more important for hydrological partitioning of precipitation than annual or seasonal totals \cite{taylor2013evidence,apurv2017understanding,singer2017deciphering,adloff-inreview}. Additionally, we see an increase in the coefficient of variation in rainfall showing that farmers experience increasingly variable rainfall conditions.

Previous studies have sparked concern over the future of rainfall in East Africa and formed a consensus about the role of climate change in influencing rainfall patterns \cite{nicholson2017climate, shongwe2011projected, adloff-inreview}. Specifically, these studies have demonstrated negative trends in the magnitude of the long rains (March-May) in East Africa and the Horn of Africa Drylands (HAD) \cite{lyon2012recent,liebmann2014understanding,williams2011westward, funk2018examining,funk2019recognizing}. Other studies have shown regional trends towards more extreme rainfall events \cite{harrison2019identifying}, particularly for the short rains (October-December) \cite{shongwe2011projected,adloff-inreview} and in central Kenya \cite{schmocker2016trends}. Adloff et al. (2020, in review) showed that the HAD received 62\% more extreme rainfall (rainfall events over the 99th percentile) for the short rains between 2003 and 2014 and 58\% more extreme rainfall in the long rains between 1981 and 2001. This further suggests that rainfall in the region is arriving in larger quantities. As we show, this trend will likely contribute to decreased end of season yields and increased rates of crop failure due to the timing and nature of rainfall and crop water requirements of maize over the season. Increasingly heavy rainfall events pose a threat to crop production due to extreme surface runoff and subsequent erosion or from flood events  \cite{liniger1998grass, liniger1998mountains}. 

We show that maize is becoming increasing difficult to grow under current climatic conditions. Small-scale producers depend on reliable rainfall. They must avoid both extreme storms (i.e. no flooding) and dry spells in order to attain end of season yields that are profitable. Our results show that despite having seasonal rainfall totals that should be adequate for maize growth, the end of season yields and chance of crop failure are highly variable. This is because the crop coefficient of maize does not align with the periods of highest rainfall when planting is conducted in early March and in these systems where irrigation may be absent. For this reason, hybrid maize has been developed to withstand variable rainfall and drought during the growing season.

%Second section
\subsection{Cultivar choice moderates exposure to stress}

We investigate the effect of cultivar choice on crop failure and yield by simulating the growth of maize varieties with early, medium and late maturation periods, which each experience differing levels of water stress over the growing season. We show that modest decreases in relative water content lead to dramatic increases in the water stress of simulated maize. When the length of the growing period is shortened for an early-maturing crop to reach the flowering and grain-filling phenological stages faster, there is a reduced chance of exposure to water deficits. Our simulated daily rainfall is governed by parameters set in dekadal (10-day) increments in which rainfall is more likely during the long rains (approx. dekads 11-14) and the short rains seasons (approx. dekads 25-34). A long-maturing 180-day variety which is planted on March 1 would be harvested at the end of August, which is just shy of the onset of the short rains. In contrast, an early-maturing variety would grow for a shortened duration and would reach critical stages of its development during the peak of the long rains season. For the 180-day variety simulations, we see the highest crop coefficient between approximately days 80 and 140; however, during this period there is a large decrease in the inter-storm arrival rate due to the cessation of the long rains. Although this period is critical for maize development as the crop has flowered and begins to grain-fill, the soil moisture content declines because of the lack of rain and the high water requirements. Because of rainfall non-stationarity within seasons, crops experience differing stress environments based on when and how long they grow. 

Cultivar choice is one of the primary adaptation strategies that farmers can control to minimize the stress experienced by the crop. We show that early-maturing crops are the best choice under lower levels of rainfall and have the lowest probability of crop failure. Late-maturing crops will produce higher yields in the simulations that do not fail, however, they have a higher risk of crop failure. Thus, early-maturing varieties are the least likely to fail just on the basis of requiring less time to grow. Furthermore, early-maturing varieties are often bred or have been genetically modified to be drought tolerant. Drought tolerance is often developed for a specific set of environmental conditions and thus can be hyper-localized. For low levels of rainfall there is a larger spread in yield outcomes for all varieties due to the pattern of rainfall within the season. As previously discussed, the intensity and timing of rainfall more than seasonal totals matters for determining outcome.

In a survey of 500 East African farming households, \citeA{erenstein2011characterization} found that the most desired attribute of maize varieties was yield potential followed by early maturity. These characteristics were perceived to be more important than drought tolerance. Maize cultivars can exhibit one of two traits: drought avoidance as in the case of early-maturing varieties and drought tolerance in the case of hybrids that are bred to tolerate reduced available water. Early-maturing varieties are considered drought avoidant because they are thought to complete their most drought-sensitive stage (flowering) before a drought occurs such as at the ending of a growing season \cite{barron2003dry, morris2001assessing}.

We show that by growing a late-maturing variety, the crop is still subject to terminal drought and crop failure due to the timing of rainfall throughout the season. However, a trade-off exists in selecting a late-maturing variety. As shown in the empirical data of maize cultivars and yield (Figure \ref{fig:ksc}), there is an opportunity for higher yields for late-maturing maize varieties. In cases where farmers who do not have access to irrigation and would prefer a crop that has a greater chance of success despite a potential yield penalty, they may prefer early-maturing crops \cite{barron2003dry}.

There are two possibilities that plant breeders and farmers can elect for: to optimize for survival by minimizing the variance of stress experienced by the crop or to optimize for greater biomass in order to get the maximum yield. In light of this, farmers may be interested in maximizing their yield as well as limiting crop failure by planting a medium-maturing variety. However, in areas with high rainfall variability there might be reason to plant only early-maturing varieties either with the intent of minimizing crop failure or in order to double crop within the season. In these settings where a farmer’s goal is to prevent crop failure, early- and extra early-maturing varieties are more effective for a planting date of the first dekad of March 1 as described in our model.  There may be a role for varieties with other maturity durations when using different planting dates. 

\subsection{Declining yields, increasing crop failure rates and household-level impacts}
% Yield and crop failure rates in the context of climatological trends

The increased variability in rainfall has serious implications for agriculture in East Africa and other regions of the African continent where large portions of the population suffer from food insecurity \cite{funk2009declining}. Even if the mean annual rainfall does not change significantly, changes to the pattern of rainfall within the season may make it difficult for farmers to practice agriculture as they have done so in the past. As shown in the temporal analysis of rainfall trends, the timing and distribution of rainfall has changed within one to two generations.

Climate change and climate variability impacts can shock the economic system by altering food prices and so affecting food demand, nutrition, and human livelihoods \cite{herrero2010climate}. We have already shown that season-to-season variability in rainfall is high, and the distribution of rainfall in addition to the total is of paramount importance to Kenyan agriculture. In Kenya, declines in per capita maize production have been reported for certain regions \cite{funka2018a}. Much of this change is due to variable seasonal rainfall and the incidence of crop failure.  Our results are consistent with work that projects reduced rainfed maize yields in Kenya \cite{herrero2010climate, thornton2010adapting}. While yield gains have been projected for certain highland areas in the temperate areas of Kenya \cite{thornton2010adapting}, farmers in the semiarid and arid lowlands are predicted to experience diminished yields, likely forcing them to sow varieties with shorter maturity periods. 

In order to capitalize on any potential yield increases in the highlands and reduce as much as possible the decreased yields in the majority of Kenya, further investment in new varieties, improved inputs, and services will be needed \cite{herrero2010climate, Hansen2011-bk}. Access to irrigation and water harvesting more generally will be an important way for farmers to buffer negative climate impacts. In this region in central Kenya, access to irrigation resources is not ubiquitous and even those farmers with access to irrigation experience high spatial and temporal variability in its availability \cite{gower2016modeling}. \citeA{mccord2018assessing} show that farmers in the Mount Kenya region with greater relative variability in water flow from irrigation are more likely to uptake adaptation measures such as choosing new seed varieties. This is a positive indication that perhaps the agriculturalists who are more impacted by rainfall variability due to reduced irrigation access are likely to employ adaptation measures or at least experiment with those measures such as changing to an early- or extra-early maturing varieties \cite{mccord2018assessing}. 

\subsection{Study Limitations and Future Research}
We made some important assumptions in our model, a common practice in such studies \cite <e.g.>[]{challinor2009crops, tesfaye2016targeting}. First, other than the rainfall climatology and the crop coefficient all other variables were constant. We do not simulate the effect of inputs such as fertilizer application or irrigation use. The model assumes that nutrients like nitrogen are available in adequate quantities that do not limit growth crop and yield. Second, we do not simulate other crops which might be intercropped with maize (e.g. beans, potatoes), and we do not simulate crop rotation. Additional studies may begin with any of these assumptions to further evaluate their effects on maize production.

Future studies would benefit from adding empirical agronomic or decision-making data as inputs or validation datasets. Our study has shown that available water content is lowest during the latter part of the season when the crop coefficient is the highest and the rainfall slackens. A follow-up study could investigate the intra-seasonal nature of water stress to demonstrate what duration of stress during the season makes the largest impact in terms of yield. To answer this question, an empirical dataset is needed that includes both intra-seasonal stress dynamics and end of season yields. Additionally, we did not constrain the behavior of early-, medium-, and late-maturing varieties other than changing the length of their growing periods. Early-maturing maize is often bred or genetically modified to be drought tolerant and thus should reduce the probability of crop failure. These constraints could be added as parameters in the dynamic water stress calculation or by altering the crop coefficient for hybrid maturities. Field-collected data or on-farm trials would be needed to constrain these parameters.